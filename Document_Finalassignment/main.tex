% Created by Bonita Graham
% Last update: December 2019 By Kestutis Bendinskas

% Authors: 
% Please do not make changes to the preamble until after the solid line of %s.

\documentclass[10pt]{article}
\usepackage[explicit]{titlesec}
\setlength{\parindent}{0pt}
\setlength{\parskip}{1em}
\usepackage{hyphenat}
\usepackage{ragged2e}
\RaggedRight

% These commands change the font. If you do not have Garamond on your computer, you will need to install it.
\usepackage{garamondx}
\usepackage[T1]{fontenc}
\usepackage{amsmath, amsthm}
\usepackage{graphicx}

% This adjusts the underline to be in keeping with word processors.
\usepackage{soul}
\setul{.6pt}{.4pt}


% The following sets margins to 1 in. on top and bottom and .75 in on left and right, and remove page numbers.
\usepackage{geometry}
\geometry{vmargin={1in,1in}, hmargin={.75in, .75in}}
\usepackage{fancyhdr}
\pagestyle{fancy}
\pagenumbering{gobble}
\renewcommand{\headrulewidth}{0.0pt}
\renewcommand{\footrulewidth}{0.0pt}

% These Commands create the label style for tables, figures and equations.
\usepackage[labelfont={footnotesize,bf} , textfont=footnotesize]{caption}
\captionsetup{labelformat=simple, labelsep=period}
\newcommand\num{\addtocounter{equation}{1}\tag{\theequation}}
\renewcommand{\theequation}{\arabic{equation}}
\makeatletter
\renewcommand\tagform@[1]{\maketag@@@ {\ignorespaces {\footnotesize{\textbf{Equation}}} #1.\unskip \@@italiccorr }}
\makeatother
\setlength{\intextsep}{10pt}
\setlength{\abovecaptionskip}{2pt}
\setlength{\belowcaptionskip}{-10pt}

\renewcommand{\textfraction}{0.10}
\renewcommand{\topfraction}{0.85}
\renewcommand{\bottomfraction}{0.85}
\renewcommand{\floatpagefraction}{0.90}

% These commands set the paragraph and line spacing
\titleformat{\section}
  {\normalfont}{\thesection}{1em}{\MakeUppercase{\textbf{#1}}}
\titlespacing\section{0pt}{0pt}{-10pt}
\titleformat{\subsection}
  {\normalfont}{\thesubsection}{1em}{\textit{#1}}
\titlespacing\subsection{0pt}{0pt}{-8pt}
\renewcommand{\baselinestretch}{1.15}

% This designs the title display style for the maketitle command
\makeatletter
\newcommand\sixteen{\@setfontsize\sixteen{17pt}{6}}
\renewcommand{\maketitle}{\bgroup\setlength{\parindent}{0pt}
\begin{flushleft}
\sixteen\bfseries \@title
\medskip
\end{flushleft}
\textit{\@author}
\egroup}
\makeatother

% This styles the bibliography and citations.
%\usepackage[biblabel]{cite}
\usepackage[sort&compress]{natbib}
\setlength\bibindent{2em}
\makeatletter
\renewcommand\@biblabel[1]{\textbf{#1.}\hfill}
\makeatother
\renewcommand{\citenumfont}[1]{\textbf{#1}}
\bibpunct{}{}{,~}{s}{,}{,}
\setlength{\bibsep}{0pt plus 0.3ex}




%%%%%%%%%%%%%%%%%%%%%%%%%%%%%%%%%%%%%%%%%%%%%%%%%

% Authors: Add additional packages and new commands here.  
% Limit your use of new commands and special formatting.

% Place your title below. Use Title Capitalization.
\title{Schedulability test for a jobset using Deadline Monotonic algorithm}

% Add author information below. Communicating author is indicated by an asterisk, the affiliation is shown by superscripted lower case letter if several affiliations need to be noted.
\author{
Youri Klaassens$^{a}$, Nick van Endhoven$^{b}$ \\ \medskip 
$^{a}$student Computer Engineering Rotterdam University of Applied Science, 0996211@hr.nl, Zwaag \\ 
$^{b}$student Computer Engineering Rotterdam University of Applied Science, 1234567@hr.nl, Breda\\ \medskip 
}

\pagestyle{empty}
\begin{document}

% Makes the title and author information appear.
\vspace*{.01 in}
\maketitle
\vspace{.12 in}

% Introduction.
\section*{introduction}

% Keywords are required.
\section*{acknowledgements}
%Note correct LaTeX quotations above. Do not use the " symbol, but rather double ` followed by double '

\begin{thebibliography}{9} %If you have more than 9 sources listed, replace this "9" with "99".

\bibitem{journal} Marquez, V., Frohlich, T., Armache, J. P., Sohmen, D., Donhofer, A., Mikolajka, A., Berninghausen, O., Thomm, M., Beckmann, R., Arnold, G. J., and Wilson, D. N. (2011) Proteomic characterization of archaeal ribosomes reveals the presence of novel archaeal-specific ribosomal proteins, \textit{J Mol Biol} 405, 1215--1232. \textit {https://doi.org/10.33697/ajur.2019.003}


\bibitem{chapters} Fierke, C. A., and Hammes, G. G. (1996) Transient Kinetic Approaches to Enzyme Mechanisms, in \textit{Contemporary Enzyme Kinetics and Mechanism} (Purich, D., Ed.) 2nd ed., 1--35, Academic Press, New York.

\bibitem{web} Agricultural Research Service, U.S.D.A. National Nutrient Database for Standard Reference, Release 26, \textit{http://ndb.nal.usda.gov/ndb/search/list} (accessed Mar 2014)
\end{thebibliography}

% The About the Student Author section is NOT optional.  Write a paragraph about the student; see previous journal editions for examples.
% If there is more than one student author, you must move the comment below.
\section*{about the student author}
%\section*{about the student authors}
Youri Klaassens is a final year bachelor student at Inholland University of Applied Science studying Computer Engineering.

Nick van Endhoven is a final year bachelor student at Avans University of Applied Science studying Computer Engineering.

% The Press Summary section is NOT optional.  Write a paragraph describing the paper in a manner suitable for the press; see previous journal editions for examples.
\section{Summary}

\end{document}
