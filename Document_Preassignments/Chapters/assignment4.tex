\section{A basic Real-Rime Operating System and \enquote{clever} scheduler}

A basic Real-Time Operating System and schudler can now be written.
In fact, the basic Real-Time Operating System has already been developed by Daniel Versluis.
From now on in this document we call this basic Real-Time Operating System \enquote{VersdOS}.
This RTOS has a global variable \texttt{taskList} which is an array of type \texttt{task}.
This datatype \texttt{task} can be seen in Listing \ref{lst:versd_task}.
Every task has its own stack and thus a pointer to the stack.
A task has a function pointer pointing to the function it should be executing if the task is ready to be run.
Also described in the previous section, every task has a state. VersdOS acknowledges three states. 
The first one is \texttt{RUNNING}, the second one is \texttt{READY} and the last one is \texttt{WAITING}.
\texttt{uiPriority} is the priority of the task. The higher this number the higher the priority.
\texttt{uiCounter} is a variable containing a number. If a task is in the \texttt{WAITING} state this number is decremented each SysTick interrupt.

\begin{lstlisting}[style=CStyle, caption={Task control block for VersdOS}, captionpos=b, label={lst:versd_task}, escapechar=@]
typedef struct _task{
	int * 			stack; 				//pointer to the stack (on the heap?)

	void(*function)(void);				//function to execute
	enum taskState	state;				//Tasks have a state
	uint8_t 		uiPriority;			//Tasks have a priority
	uint32_t		uiCounter;			//Tasks have a counter for delays

	char			bInitialized;		//After initialization the stack also contains R4-R11
} task;
\end{lstlisting}

The current implementation of the core functionality of the scheduler, which is selecting the next task to run, can be seen in Listing \ref{lst:versd_old_sched}.
There are three tasks. One that toggles the red LED, one that toggles the green LED and one that toggles the yellow LED.
The scheduler is aware of the amount of tasks but its hard coded. This is very bad practice and makes a scheduler only usable for a certain set of task jobs \cite{IntroEmbeddedRtosSystems}.
The current implementation is ignoring the \texttt{uiPriority} member variable of the task control block aswell.

\begin{lstlisting}[style=CStyle, caption={A relatively stupid scheduler}, captionpos=b, label={lst:versd_old_sched}, escapechar=@]
/*
 * Our genuine scheduler. Currently very static and
 * not-advanced
 *
 * returns a pointer to the selected task
 */
task * getNextTask(void)
{
	static int i=0;

	while(taskList[++i].function == 0)
	{
		if(i==3)
		{
			i = -1;
		}
	}

	return &taskList[i];
}
\end{lstlisting}

This scheduler in Listing \ref{lst:versd_old_sched} is not real time as it is not taking the characteristics or deadlines of any task into account.
A better implementation in terms of guarantees and taking real-time characteristics into account is proposed in Listing \ref{lst:versd_new_sched}.

\newpage
\begin{lstlisting}[style=CStyle, caption={An improved scheduler taking task characteristics into account}, captionpos=b, label={lst:versd_new_sched}, escapechar=@]
/*
 * Our genuine scheduler. Currently very static and
 * not-advanced
 *
 * returns a pointer to the selected task
 */
task * getNextTask(void)
{
    int i = 0;

    int idx_task = 0;
    int task_priority_temp = 0;

    for(i = 0; i < amount_of_tasks; i++)
    {
	    if(taskList[i].state != WAITING)
	    {
            if(taskList[i].uiPriority > task_priority_temp)
            {
                idx_task = i;
                task_priority_temp = taskList[i].uiPriority;
            }
        }
    }

    return &taskList[idx_task]; @\label{line:versd_idle_select}@
}
\end{lstlisting}

One can argue \enquote{what happends if all tasks are \texttt{WAITING}?}
\texttt{taskList[0]} always contains the idle task.
That is why variable \texttt{idx\_task} is set to 0 before the for-loop starts.
If no other tasks wants to use the micrcoprocessor then the idle task will be returned in Line \ref{line:versd_idle_select}.
This idle task can be seen in Listing \ref{lst:versd_idle_task}.
All it does is setting the microprocessor in a sleep mode untill an event occurs.

\begin{lstlisting}[style=CStyle, caption={An idle task basically setting the microprocesor in a sleep mode}, captionpos=b, label={lst:versd_idle_task}, escapechar=@]
void idleTask(void)
{
    __asm("    wfi");
}
\end{lstlisting}


Another thing that the authors noticed is the current implementation of a delay function to temporarry halt a certain task.
The old implementation can be seen in Listing \ref{lst:versd_old_delay}.
This delay function depends on the clock frequency and compiler generated instructions.
In addition, a task is not sleeping. IT is doing a function call to \texttt{delay} and hopes when the function returns it slept long enough.
It keeps sitting on the CPU executing dumb \texttt{SUBS} instructions.
Other tasks in the \texttt{READY} state cannot claim the CPU as the \texttt{RUNNING} task keeps the CPU busy by doing meaningless calculations (if it calls the \texttt{delay} functions).
The authors agree with Daniel Versluis that this is an unworthy \texttt{delay} function.
\begin{lstlisting}[style=CStyle, caption={An unworthy delay function, according to Daniel Versluis}, captionpos=b, label={lst:versd_old_delay}, escapechar=@]
//unworthy delay function
void delay(unsigned int count)
{
    while(count--);
}

\end{lstlisting}

The authors propose a worthy \texttt{delay} function. The implementation details can be seen in Listing \ref{lst:versd_new_delay}.


\newpage
\begin{lstlisting}[style=CStyle, caption={A worhy \texttt{delay} function, according to the authors}, captionpos=b, label={lst:versd_new_delay}, escapechar=@]
void delayTask(unsigned int ticks)
{
    for(int idx = 0; idx < MAX_TASKS; idx++)    @\label{line:versd_search_running}@
    {
        if(taskList[idx].state == RUNNING){     @\label{line:versd_is_running}@
            taskList[idx].uiCounter = ticks;
            taskList[idx].state = WAITING;
        }
    }
    schedule();     @\label{line:versd_delay_call_schedule}@
}
\end{lstlisting}

The proposed \texttt{delay} deserves some explanation as it may not seem trivial in the first place.
This implementation makes use of a property of a single core microprocessor.
Only one task at a certain point in time can have the \texttt{RUNNING} state.
If the function \texttt{delay} is entered it must be called from the current \texttt{RUNNING} task.
It assumes the function call is atomic and no scheduling happends when between the function is called and the state of the running task is set to \texttt{WAITING}.
The loop in Line \ref{line:versd_search_running} searches through the list of registered tasks.
If the condition in Line \ref{line:versd_is_running} is true it means that the \texttt{RUNNING} task is found.
If it is found the \texttt{uiCounter} variable which is decremented each \texttt{SysTickHandler} interrupt is decremented is set to the desired amount of ticks passed via the function parameter.
The \texttt{state} is set to \texttt{WAITING}.
The housekeeping is set. The task is ready to be put in \texttt{WAITING} mode.
1 important detail remains.
A new task should be selected to be run on the microprocessor.
The function call in Line \ref{line:versd_delay_call_schedule} will take care of this.
This function \texttt{schedule} can be seen in Listing \ref{lst:versd_new_schedule}.

\begin{lstlisting}[style=CStyle, caption={\texttt{schedule} will call the helper function \texttt{getNextTask} and makes sure that task is actually executed}, captionpos=b, label={lst:versd_new_schedule}, escapechar=@]
void schedule(void)
{
	//This is our true scheduler function
	//select a new task to run
	taskToExecute = getNextTask();

	//Only if the new task isn't equal to the current one,
	//call the context switch
	if(taskToExecute != currentTask || taskToExecute->bInitialized==0)
	{
		//States to help the scheduler decide
		//currentTask->state = READY;
		taskToExecute->state = RUNNING;

		//call pendsv interrupt to perform the context switch
		HWREG(NVIC_INT_CTRL)  |= (1<<28);

	}else{
		//Clearly no need to switch anything so we
		//just restore things like they were before the Systick

		currentTask->state	 = RUNNING;
	}
}
\end{lstlisting}

Because of lack of time there was no time to show prove using the logic analyzer that this implementation works.
Various instructors have seen during labs that it works and signed off the authors name on a list of assignments.
If the reader still wants proof then he or she must send an e-mail to one of the authors e-mail addresses on the title page.


