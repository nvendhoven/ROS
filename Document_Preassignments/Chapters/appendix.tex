\section{Appendix}

The appendix contains subsections that support this report or its where its content goes too much off-topic with the purpose of 
this report, but are interesting for the reader to possibly read.

\subsection{Delay \textit{exactly} one second counting instruction cycles}

Many assignments require a delay of 1 second to spot blinky LEDs by eye.
One can use the Systick timer or hardware timers, but where is the fun in that?
For the sake of some assignments, it is acceptable to burn clock cycles by wasting the CPU.
Listing \ref{lst:delay1sec} contains a function which will delay the return moment by 1 second.
Now each line containing inline assembly will be explained.

\begin{lstlisting}[style=CStyle, caption={C function containing inline assembly to perform a delay of \textit{exactly} one second}, captionpos=b, label={lst:delay1sec}, escapechar=|]
void delay_1sec(void)
{
    __asm("    PUSH {r4-r11,lr}");  |\label{line:delay1sec_push}|
 
    __asm("    LDR r4, [pc, #12]");
   
    __asm("    MOV r5, pc");
    __asm("    NOP");
     
    __asm("    SUBS r4, #1");   /* 1 instruction cycle */
    __asm("    ITE NEQ");       /* 1 instruction cycle */
   
    __asm("    MOV pc, r5");    /* 1 + P instructions (where P is between 1 and 3 depending on pipeline refill) */
     
     
    __asm("    POP {r4-r11,pc}");
    __asm("    .word    0x5000000");
}
\end{lstlisting}

Line \ref{line:delay1sec_push} pushes 8 registers onto the stack. 
This is part of the ARM Architecture Procedure Call Standard (AAPCS) which is part of the ARM Application Binary Interface (ABI). \cite{IntroEmbeddedSystems}
This standard describes that \texttt{R0} up to and including \texttt{R4} are used to pass input parameters into a C function. 
Functions should preserve the content of registers \texttt{R4} up to and including \texttt{R11}.
Listing \ref{lst:delay1sec} does not use all of the registers a callee should save, but it is best practice to push them in case one does not know how many registers his or her piece of software will use.

