\section{Appendix}

The appendix contains subsections that support this report or its where its content goes too much off-topic with the purpose of 
this report, but are interesting for the reader to possibly read.

\subsection{Delay \textit{exactly} one second counting instruction cycles}
\label{subsec:appendix_delay}

Many assignments require a delay of 1 second to spot blinky LEDs by eye.
One can use the Systick timer or hardware timers, but where is the fun in that?
For the sake of some assignments, it is acceptable to burn clock cycles by wasting the CPU.
Listing \ref{lst:delay1sec} contains a function which will delay the return moment by 1 second.
Now each line containing inline assembly will be explained.

\begin{lstlisting}[style=CStyle, caption={C function containing inline assembly to perform a delay of \textit{exactly} one second}, captionpos=b, label={lst:delay1sec}, escapechar=|]
void delay_1sec(void)
{
    __asm("    PUSH {r4-r11,lr}");  |\label{line:delay1sec_push}|
 
    __asm("    LDR r4, [pc, #12]"); |\label{line:delay1sec_getword}|
   
    __asm("    MOV r5, pc");        |\label{line:delay1sec_storepc}|
    __asm("    NOP");               |\label{line:delay1sec_nop}|
     
    __asm("    SUBS r4, #1");   /* 1 instruction cycle */ |\label{line:delay1sec_sub}|
    __asm("    ITE NEQ");       /* 1 instruction cycle */ |\label{line:delay1sec_ite}|
   
    __asm("    MOV pc, r5");    /* 1 + P instructions (where P is between 1 and 3 depending on pipeline refill) */ |\label{line:delay1sec_restorepc}|
     
     
    __asm("    POP {r4-r11,pc}"); |\label{line:delay1sec_pop}|
    __asm("    .word    5000000"); |\label{line:delay1sec_word}|
}
\end{lstlisting}

Line \ref{line:delay1sec_push} pushes 8 registers onto the stack. 
This is part of the ARM Architecture Procedure Call Standard (AAPCS) which is part of the ARM Application Binary Interface (ABI) \cite{IntroEmbeddedSystems}.
This standard describes that \texttt{R0} up to and including \texttt{R4} are used to pass input parameters into a C function. 
Functions should preserve the content of registers \texttt{R4} up to and including \texttt{R11}.
Listing \ref{lst:delay1sec} does not use all of the registers a callee should save, but it is best practice to push them in case one does not know how many registers his or her piece of software will use.\\
%TODO: Explain line two
Line \ref{line:delay1sec_storepc} stores the Program Counter (PC) into \texttt{R5}. 
Because the PC is two instruction (8 bytes) ahead in ARM mode it actually stores the address for Line \ref{line:delay1sec_sub}.
This is the first instruction that should be executed iterative.
Line \ref{line:delay1sec_nop} makes sure the instruction located at Line \ref{line:delay1sec_storepc} contains the correct address. The alternative is replacing this instruction with a \texttt{SUB} instruction and subtract 4 bytes from \texttt{R5}.
Line \ref{line:delay1sec_ite} does a check whether the content of \texttt{R4} is equal to zero or not \cite{DefinitiveGuide}.
If \texttt{R4} is not equal to zero (which makes the statement true because we check for \texttt{NEQ} condition code) Line \ref{line:delay1sec_restorepc} is executed. If \texttt{R4} is equal to zero Line \ref{line:delay1sec_pop} is executed.
Line \ref{line:delay1sec_restorepc} stores the PC we saved earlier in Line \ref{line:delay1sec_storepc} to the PC.
This results a branch to Line \ref{line:delay1sec_sub}.
Line \ref{line:delay1sec_pop} restores the saved registers and jumps back to the caller. 
It is not an option to leave out the restore to the PC because that means that the next instruction executed will be the one on Line \ref{line:delay1sec_word}.
This is not an intentional instruction but just a location to store a number. If we let the PC execute this line we get undefined behaviour.

\newpage
\subsection{Events and the vector table in the ARM Cortex devices}
\label{subsec:appendix_vector}

The Nested Vector Interrupt Controller (NVIC) is a hardware module within the processor to prioritize events.
A reader familiar with 8 bit microcontrollers may wonder why we use the term interrupts and events and if they are interchangeable.
In ARM terminology, an interrupt is one type of exception. 
Other exceptions in Cortex-M processors include fault exception an other system exceptions to support the OS (e.g., SVC instruction) \cite{DefinitiveGuide}.
Other readers  familiar with x86 can compare the NVIC to the Programmable Interrupt Controller (PIC). \newline

\begin{lstlisting}[style=CStyle, caption={Vector table used in assignment two and three}, captionpos=b, label={lst:vector_table_complete}, escapechar=|]
#pragma RETAIN(resetVectors)
#pragma DATA_SECTION(resetVectors, ".resetVecs")
void (* const resetVectors[43])(void) =
{
    (void (*)(void))((unsigned long)&__STACK_END),
                                         // The initial stack pointer
    resetISR,                            // The reset handler
    nmiISR,                              // The NMI handler
    faultISR,                            // The hard fault handler
    defaultHandler,                      // The MPU fault handler
    busFaultHandler,                     // The bus fault handler
    defaultHandler,                      // The usage fault handler
    0,                                   // Reserved
    0,                                   // Reserved
    0,                                   // Reserved
    0,                                   // Reserved
    defaultHandler,                     // SVCall handler
    defaultHandler,                      // Debug monitor handler
    0,                                   // Reserved
    defaultHandler,                      // The PendSV handler
    SysTickHandler,                      // The SysTick handler
    defaultHandler,                      // GPIO Port A0
    defaultHandler,                      // GPIO Port A1
    defaultHandler,                      // GPIO Port A2
    defaultHandler,                      // GPIO Port A3
    0,                      // Reserved
    defaultHandler,                      // UART0 Rx and Tx
    defaultHandler,                      // UART1 Rx and Tx
    0,                      // Reserved
    defaultHandler,                      // I2C0 Master and Slave
    0,                      // Reserved
    0,                      // Reserved
    0,                      // Reserved
    0,                      // Reserved
    0,                      // Reserved
    defaultHandler,                      // ADC Sequence 0
    defaultHandler,                      // ADC Sequence 1
    defaultHandler,                      // ADC Sequence 2
    defaultHandler,                      // ADC Sequence 3
    defaultHandler,                      // Watchdog timer
    defaultHandler,                      // Timer 0 subtimer A
    defaultHandler,                      // Timer 0 subtimer B
    defaultHandler,                      // Timer 1 subtimer A
    defaultHandler,                      // Timer 1 subtimer B
    defaultHandler,                      // Timer 2 subtimer A
    defaultHandler,                      // Timer 2 subtimer B
    defaultHandler,                      // Timer 3 subtimer A
    defaultHandler                      // Timer 3 subtimer B
};

\end{lstlisting}

The order of the entries in this table is important and predetermined.
Exception 1-15 are system exceptions and exception 16 and above are interrupt inputs \cite{DefinitiveGuide}.
Exception 1 is always the Reset exception, exception 2 is always the NonMaskable Interrupt (NMI), exception 3 is always Hard Fault and so on \cite{DefinitiveGuide}.
From exception 16 it is manufacturer dependent. 
It depends if manufacturers put certain hardware modules in their core or not.
It is therefore always recommended to check the datasheet of the microcontroller.

\newpage
When an event occurs there is generally a sequence consisting of 5 steps that occurs \cite{IntroEmbeddedSystems}.
\begin{enumerate}
    \item {The current instruction is finished}
    \item {The execution of the currently running program is suspended, pushing eight registers on the stack (\texttt{R0}, \texttt{R1}, \texttt{R2}, \texttt{R3}, \texttt{R12}, \texttt{LR}, \texttt{PC} and \texttt{PSR} (assuming ARMv7-M architecture)}
    \item {The \texttt{LR} is set to a specific value signifying an ISR is being run}
    \item {The \texttt{IPSR} is set to the event number being processed}
    \item {The \texttt{PC} is loaded with the address of the event handler (from the vector table)}
\end{enumerate}

The \texttt{IPSR} is the Interrupt Program Status Register and the least 9 significant bits represent the exception number being executed.
The other Program Status Registers are the Application Program Status Register (\texttt{APSR}) and the Execution Program Status Register (\texttt{EPSR}).
These will not be explained because they are too much off-topic regarding this report.
